\documentclass[11pt,letterpaper]{article}
\usepackage[left=3.5cm,right=3.5cm,top=2.5cm,bottom=2.5cm]{geometry}
%\usepackage[spanish]{babel}

\usepackage{amsmath,amsfonts,amsthm}
\usepackage{enumitem,mathtools}
\setenumerate[0]{label=(\alph*)}

\newtheorem{exercise}{Ejercicio}
\newtheorem*{sol}{Solución}

\newcommand\N{\mathbb N}
\newcommand\tol{\mathrm{tol}}

\title{Primer examen parcial - Análisis numérico para ecuaciones
diferenciales}
\author{Jorge Alfredo Álvarez Contreras}

\begin{document}
\maketitle

\begin{exercise}
  Considere el problema de valores iniciales
  \begin{equation}
    y' = \frac{y+t}{y-t}, \quad y(0)=1
  ,\end{equation}
  cuya solución es $y(t)=t+\sqrt{1+2t^{2}}$. Utilice $h=0.1$ para
  \begin{enumerate}
    \item
      realizar dos pasos del método de Euler explícito
    \item
      realizar dos pasos del método de Euler implícito (utilice
      iteración Newton-Raphson con una tolerancia de $10^{-5}$).
  \end{enumerate}
\end{exercise}
\begin{sol}
  \begin{enumerate}
    \item
      Para $h=0.1$, tenemos $t_0=0$, $t_1=h=0.1$, así que el método de
      Euler explícito
      \begin{equation}
        u_{n+1} = u_n + hf_n
      \end{equation}
      comienza con $u_0=y(0)=1$ y produce las aproximaciones
      \begin{align}
        u_1
        &= u_0+hf_0 \\
        &= u_0+h \frac{u_0+t_0}{u_0-t_0} \\
        &= 1+h \\
        &= 1.1,
        \\
        u_2
        &= u_1+hf_1 \\
        &= u_1+h \frac{u_1+t_1}{u_1-t_1} \\
        &= 1.1+h \frac{1.1+0.1}{1.1-0.1} \\
        &= 1.1+(0.1)(1.2) \\
        &= 1.1 + 0.12 \\
        &= 1.22
      .\end{align}
    \item
      Para el método de Euler implícito,
      \begin{equation}
        u_{n+1} = u_n + hf_{n+1}
      ,\end{equation}
      también tenemos $t_0=0$, $t_1=h=0.1$
      y comenzamos con $u_0=y(0)=1$.
      Así, debemos resolver las ecuaciones
      \begin{equation}
        \left\{
          \begin{aligned}
            u_1 &= u_0 + hf_{1},
            \\
            u_2 &= u_1 + hf_{2}.
          \end{aligned}
        \right.
      \end{equation}
      Esto es
      \begin{equation}
        \left\{
          \begin{aligned}
            u_1 &= u_0 + h \frac{u_1+t_1}{u_1-t_1}
            \\
            u_2 &= u_1 + h \frac{u_2+t_2}{u_2-t_2}
          \end{aligned}
        \right.
      \end{equation}
      O bien
      \begin{equation}
        \left\{
          \begin{aligned}
            u_1 - u_0 - h \frac{u_1+t_1}{u_1-t_1} &= 0
            \\
            u_2 - u_1 - h \frac{u_2+t_2}{u_2-t_2} &= 0
          \end{aligned}
        \right.
      \end{equation}
      es decir, hay que encontrar las raíces de
      \begin{equation}\label{eq:func_objetivo}
        \phi(u_n) = u_{n} - u_{n-1} - h \frac{u_n+t_n}{u_n-t_n}
      ,\end{equation}
      para $n=1,2$. Para esto usaremos el método de Newton-Raphson.
      Recordemos que este es un método iterativo para encontrar
      aproximaciones sucesivas $x^{(1)},x^{(2)},x^{(3)},\dots$ a una
      solución $x$ de la ecuación $\phi(x)=0$, comenzando
      con un valor inicial $x^{(0)}$ y definiendo
      \begin{equation}
        x^{(k+1)} = x^{(k)} - \frac{\phi(x^{(k)})}{\phi'(x^{(k)})}
      .\end{equation}
      El método se detiene cuando $|x^{(k+1)}-x^{(k)}|<\tol$.
      En este caso, la función de la cual queremos obtener raíces es
      \eqref{eq:func_objetivo},
      donde $u_{n-1}, t_n$ y $h$ son conocidas.
      Así, la derivada de $\phi$ es
      \begin{align}
        \phi'(u_n)
        &= 1 - h \frac{(u_n-t_n)-(u_n+t_n)}{(u_n-t_n)^{2}} \\
        &= 1 + \frac{2ht_n}{(u_n-t_n)^{2}}
      ,\end{align}
      y las aproximaciones sucesivas
      $u_n^{(1)},u_n^{(2)},u_{n}^{3},\dots$ se calculan a partir de
      $u_{n}^{(0)}$como
      \begin{align}
        u_{n}^{(k+1)}
        &= u_{n}^{(k)} - \frac{\phi(u_n^{(k)})}{\phi'(u_n^{(k)})} \\
        &= u_{n}^{(k)}
        - \frac
        {
          \displaystyle
          u_{n}^{(k)}
          - u_{n-1}
          - h \frac{u_n^{(k)}+t_n}{u_n^{(k)}-t_n}
        }
        {
          \displaystyle
          1
          + \frac{2ht_n}{(u_n^{(k)}-t_n)^{2}}
        } \\
        &= u_{n}^{(k)}
        - \frac
        {
          (u_n^{(k)}-t_n)^{2}(u_{n}^{(k)} - u_{n-1})
          - h ((u_n^{(k)})^{2}-t_n^{2})
        }
        {
          (u_n^{(k)}-t_n)^{2} + 2ht_n
        }
      .\end{align}
      Comencemos con $u_1$. Calcularemos nuestro valor inicial con el
      método de Euler explícito:
      \begin{equation}
        u_1^{(0)}=u_0 + h f_0 = 1 + (0.1)(1)=1.1
      .\end{equation}
      En este caso, las aproximaciones sucesivas se calculan como
      \begin{align}
        u_1^{(k+1)}
        &=
          u_{1}^{(k)}
          - \frac
          {
            (u_1^{(k)}-t_1)^{2}(u_{1}^{(k)} - u_{0})
            - h ((u_1^{(k)})^{2}-t_1^{2})
          }
          {
            (u_1^{(k)}-t_1)^{2} + 2ht_1
          }
          \\
        &=
          u_{1}^{(k)}
          - \frac
          {
            (u_1^{(k)}-0.1)^{2}(u_{1}^{(k)} - 1)
            - (0.1) ((u_1^{(k)})^{2}-0.01)
          }
          {
            (u_1^{(k)}-0.1)^{2} + 0.02
          }
      .\end{align}
      Usando el valor inicial $u_1^{(0)}=1.1$, tenemos, a 6
      decimales,
      \begin{align}
        u_1^{(1)}
        &=
          u_1^{(0)}
          - \frac
          {
            (u_1^{(k)}-0.1)^{2}(u_{1}^{(k)} - 1)
            - (0.1) ((u_1^{(k)})^{2}-0.01)
          }
          {
            (u_1^{(k)}-0.1)^{2} + 0.02
          }
          \\
        &= u_1^{(0)} - \frac { 0.1 - 0.12 } { 1.02 } \\
        &\approx
          u_1^{(0)} + 0.019608 \\
        &= 1.119608
      .\end{align}
      Notemos que $|u_{1}^{(1)}-u_1^{(0)}| = 0.019608 > \tol =
      0.00001$, así que calculamos otra iteración. A 6 decimales,
      tenemos
      \begin{align}
        u_1^{(2)}
        &=
          u_{1}^{(1)}
          - \frac
          {
            (u_1^{(1)}-0.1)^{2}(u_{1}^{(1)} - 1)
            - (0.1) ((u_1^{(1)})^{2}-0.01)
          }
          {
            (u_1^{(1)}-0.1)^{2} + 0.02
          }
          \\
        &=
          u_{1}^{(1)} - \frac { 0.124345 - 0.124352} { 1.059600 }
          \\
        &\approx u_{1}^{(1)} + 0.000007 \\
        &= 1.119615
      .\end{align}
      La diferencia es
      $|u_{1}^{(2)}-u_1^{(1)}|=0.000007<\tol=0.00001$, así que
      podemos aceptar el valor $u_1^{(2)}=1.119615$ como $u_1$.

      Ahora aproximaremos $u_2$. Para estimar el valor inicial,
      usaremos un paso del método de Euler explícito:
      \begin{align}
        u_2^{(0)}
        &= u_1 + hf_1 \\
        &= u_1 + h \frac{u_1+t_1}{u_1-t_1} \\
        &= 1.119615 + (0.1)\frac{1.219615}{1.019615} \\
        &\approx 1.239230
      ,\end{align}
      
      Las aproximaciones sucesivas de $u_2$ 
      se calculan como
      \begin{align}
        u_2^{(k+1)}
        &=
          u_{2}^{(k)}
          - \frac
          {
            (u_2^{(k)}-t_2)^{2}(u_{2}^{(k)} - u_{1})
            - h ((u_2^{(k)})^{2}-t_2^{2})
          }
          {
            (u_2^{(k)}-t_2)^{2} + 2ht_2
          }
          \\
        &=
          u_{2}^{(k)}
          - \frac
          {
            (u_2^{(k)}-0.2)^{2}(u_{2}^{(k)} - 1.119615)
            - (0.1)((u_2^{(k)})^{2}-0.04)
          }
          {
            (u_2^{(k)}-0.2)^{2} + 0.04
          }
      .\end{align}
      Calculando hasta 6 decimales, tenemos
      \begin{align}
        u_2^{(1)}
        &=
          u_{2}^{(0)}
          - \frac
          {
            (u_2^{(0)}-0.2)^{2}(u_{2}^{(0)} - 1.119615)
            - (0.1)((u_2^{(0)})^{2}-0.04)
          }
          {
            (u_2^{(0)}-0.2)^{2} + 0.04
          }
          \\
        &\approx
          u_{2}^{(0)} - \frac { 0.129184 - 0.149569} { 1.12 } \\
        &\approx u_{2}^{(0)} + 0.018201 \\
        &= 1.257431
      .\end{align}
      Luego, $|u_2^{(1)}-u_2^{(0)}|=0.018201>\tol=0.00001$, así que
      calculamos otra iteración.
      \begin{align}
        u_2^{(2)}
        &=
          u_{2}^{(1)}
          - \frac
          {
            (u_2^{(1)}-0.2)^{2}(u_{2}^{(1)} - 1.119615)
            - (0.1)((u_2^{(1)})^{2}-0.04)
          }
          {
            (u_2^{(0)}-0.2)^{2} + 0.04
          }
          \\
        &\approx
          u_{2}^{(0)} - \frac { 0.154100 - 0.154113 } { 1.158160 } \\
        &\approx u_{2}^{(0)} + 0.000011 \\
        &= 1.257442
      .\end{align}
      Tenemos $|u_2^{(2)}-u_2^{(1)}|=0.000011>\tol=0.00001$.
      Calculamos otra iteración:
      \begin{align}
        u_2^{(3)}
        &=
          u_{2}^{(2)}
          - \frac
          {
            (u_2^{(2)}-0.2)^{2}(u_{2}^{(2)} - 1.119615)
            - (0.1)((u_2^{(2)})^{2}-0.04)
          }
          {
            (u_2^{(0)}-0.2)^{2} + 0.04
          }
          \\
        &\approx
          u_{2}^{(0)} - \frac { 0.154116 - 0.154116 } { 1.158184 } \\
        &\approx u_{2}^{(0)} + 0.000000 \\
        &= 1.257442
      .\end{align}
      Así, $|u_2^{(3)}-u_2^{(2)}|=0.000000<\tol=0.00001$. Luego,
      podemos aceptar $u_2^{(3)}=1.257442$ como valor para $u_2$.
  \end{enumerate}
  Los resultados obtenidos se resumen en la siguiente tabla.
  \begin{equation}
    \begin{array}{c|c|c|c}
      & u_0 & u_1 & u_2 \\
      \hline
      \text{Euler explícito}
        & 1 & 1.1 & 1.22 \\
      \hline
      \text{Euler implícito + Newton-Raphson}
        & 1 & 1.119615 & 1.257442
    \end{array}
  \end{equation}
\end{sol}

\begin{exercise}
  Considere el problema de valores iniciales
  \begin{equation}
    \left\{
      \begin{aligned}
        y' &= -y
        \\
        y(0) &= 1
      \end{aligned}
    \right.
  \end{equation}
  Sean $h=\frac{1}{2}$ y $u_1=\frac{1}{2}$. Aplique el método
  $u_{n+2}=u_n + 2hf(t_{n+1},u_{n+1})$ y determine una fórmula
  explícita para $u_n$. Muestre que $u_n$ diverge cuando $n\to\infty$.
\end{exercise}
\begin{sol}
  Dado que $f(t,y)=-y$, el método se reduce a
  \begin{equation}
    u_{n+2}=u_n - 2hu_{n+1}
  .\end{equation}
  Considerando soluciones de la forma $u_n=r^{n}$, con $r\neq 0$,
  tenemos
  \begin{equation}
    r^{2} = 1 - 2hr
  .\end{equation}
  Dado que $h=\frac{1}{2}$, esto es $r^{2}+r-1=0$, así que las
  soluciones son
  \begin{align}
    r_{1,2} &= \frac{-1 \pm \sqrt{5}}{2}
  ,\end{align}
  Por lo tanto, la solución general es de la forma
  \begin{equation}
    u_n = \gamma_1 
    \left(\frac{-1 - \sqrt{5}}{2}\right)^{n}
    + \gamma_2
    \left(\frac{-1 + \sqrt{5}}{2}\right)^{n}
  .\end{equation}
  Así, para satisfacer las condiciones iniciales $u_0=y(0)=1$ y
  $u_1=\frac{1}{2}$, debemos resolver el sistema
  \begin{equation}
    \left\{
      \begin{aligned}
          1
          &= \gamma_1 
          + \gamma_2
        \\
          \frac{1}{2}
          &= \gamma_1 
          \frac{-1 - \sqrt{5}}{2}
          + \gamma_2
          \frac{-1 + \sqrt{5}}{2}
      \end{aligned}
    \right.
  \end{equation}
  Cancelando los denominadores de la segunda ecuación, tenemos
  \begin{align}
    1
    &= \gamma_1 (-1 - \sqrt{5}) + \gamma_2 (-1 + \sqrt{5}) \\
    &= -\gamma_1 -\gamma_1  \sqrt{5} -\gamma_2 +\gamma_2  \sqrt{5} \\
    &= -(\gamma_1+\gamma_2) -(\gamma_1-\gamma_2) \sqrt{5}
  .\end{align}
  Ahora, usando la primera ecuación, tenemos $\gamma_1+\gamma_2=1$ y
  $\gamma_1-\gamma_2=1-2\gamma_2$, así que
  \begin{equation}
    1 = -1 -(1-2\gamma_2) \sqrt{5}
  .\end{equation}
  Luego,
  \begin{equation}
     \gamma_2
     = \frac{1}{2} + \frac{1}{\sqrt 5}
     = \frac{\sqrt 5 + 2}{2\sqrt 5}
  \end{equation}
  y, por lo tanto,
  \begin{equation}
    \gamma_1
    = \frac{1}{2}- \frac{1}{\sqrt 5}
     = \frac{\sqrt 5 - 2}{2\sqrt 5}
  ,\end{equation}
  Así, obtenemos una fórmula explícita para los $u_n$.
  \begin{align}
    u_n
    &=
    \frac{\sqrt 5 - 2}{2\sqrt 5}
    \frac{(-1)^{n}(1 + \sqrt{5})^{n}}{2^n}
    + \frac{\sqrt 5 + 2}{2\sqrt 5}
      \frac{(-1)^{n}(1 - \sqrt{5})^{n}}{2^n}
      \\
    &=
    \frac{(-1)^{n}}{2\sqrt 5}
    \left[
    (\sqrt 5 - 2)
    \frac{(1 + \sqrt{5})^{n}}{2^n}
    + (\sqrt 5 + 2)
      \frac{(1 - \sqrt{5})^{n}}{2^n}
      \right]
  .\end{align}
  Nótese que los primeros valores de la sucesión son
  \begin{equation}
    \frac{2}{2}, \frac{1}{2}, \frac{1}{2}, 0, \frac{1}{2},
    -\frac{1}{2}, \frac{2}{2},
    -\frac{3}{2}, \frac{5}{2}, -\frac{8}{2}, \frac{13}{2},
    -\frac{21}{2}, \frac{33}{2}, \dots
  .\end{equation}
  Así, parece que la sucesión $u_n$ es tan solo la sucesión de
  Fibonacci, desplazada, con signos alternantes y dividida entre dos.
  Podemos demostrar esto notando que
  \begin{equation}
    \left(
      \frac{1\pm\sqrt 5}{2}
    \right)^{3}
    = 2 \pm \sqrt 5
  .\end{equation}
  Por lo tanto, tenemos
  \begin{align}
    u_{n+3}
    &=
      \frac{(-1)^{n+3}}{2\sqrt 5}
      \left[
      (\sqrt 5 - 2)
      \frac{(1 + \sqrt{5})^{n+3}}{2^{n+3}}
      + (\sqrt 5 + 2)
      \frac{(1 - \sqrt{5})^{n+3}}{2^{n+3}}
      \right]
      \\
    &=
      \frac{(-1)^{n+1}}{2\sqrt 5}
      \left[
      (\sqrt 5 - 2)
      (2+\sqrt 5)
      \frac{(1 + \sqrt{5})^{n}}{2^{n}}
      + (\sqrt 5 + 2)
      (2-\sqrt 5)
      \frac{(1 - \sqrt{5})^{n}}{2^{n}}
      \right]
      \\
    &=
      \frac{(-1)^{n+1}}{2\sqrt 5}
      \left[
      \frac{(1 + \sqrt{5})^{n}}{2^{n}}
      -
      \frac{(1 - \sqrt{5})^{n}}{2^{n}}
      \right]
      \\
    &=
      \frac{(-1)^{n+1}}{2} F_n,
  \end{align}
  donde $F_n$ es el $n$-ésimo número de Fibonacci, dado por
  \begin{equation}
    F_n = 
      \frac{1}{\sqrt 5}
      \left[
      \frac{(1 + \sqrt{5})^{n}}{2^{n}}
      -
      \frac{(1 - \sqrt{5})^{n}}{2^{n}}
      \right]
  .\end{equation}
  Dado que sabemos que los números de Fibonacci crecen
  indefinidamente, concluimos que
  \begin{equation}
    |u_{n+3}| = \frac{1}{2}|F_n| \to \infty
  ,\end{equation}
  cuando $n\to\infty$.
\end{sol}

\begin{exercise}
  Determine el error de truncamiento local $\tau_{n+1}(h)$ para el
  método de punto medio dado por
  \begin{equation}
    u_{n+1} = u_n + hf \left(
      t_n + \frac{h}{2} , u_n + \frac{h}{2} f_n
    \right)
  .\end{equation}
  Adicionalmente, considere el problema de valores iniciales $y'=-2t$,
  $y(0)=0$. Muestre que, cuando el método es aplicado a dicho
  problema, este produce la solución exacta.
\end{exercise}

\begin{sol}
  Para calcular el error de truncamiento local, consideremos primero
  el error residual $\epsilon_{n+1}$, que se define por la ecuación:
  \begin{equation}
    y_{n+1} = y_n + h f\left(t_n + \frac{h}{2}, y_n +
    \frac{h}{2}f(t_n,y_n)\right) + \epsilon_{n+1}
  .\end{equation}
  Es decir,
  \begin{equation}
    \epsilon_{n+1}
    =
    y_{n+1} - y_n - h f\left(t_n + \frac{h}{2}, y_n + \frac{h}{2}f(t_n,y_n)\right)
  ,\end{equation}
  de modo que
  \begin{equation}\label{eq:etl}
    \tau_{n+1}(h)
    = \frac{\epsilon_{n+1}}{h}
    = \frac{y_{n+1} - y_n}{h}
    - f\left(t_n + \frac{h}{2}, y_n + \frac{h}{2}y'_n\right)
  ,\end{equation}
  donde sustituimos $y'_n = f(t_n,y_n)$. Dado que también tenemos
  \begin{equation}
    y''_n = Df(t_n,y_n)((1,y'_n))
  ,\end{equation}
  podemos expandir la ecuación \eqref{eq:etl} hasta orden $2$ 
  en $h$ y obtener
  \begin{align}
    \tau_{n+1}(h) 
    &=
    y'_n + \frac{h}{2}y''_n + \frac{h^{2}}{6}y'''_n
    - f\left(t_n + \frac{h}{2}, y_n + \frac{h}{2}y'_n\right) +
    O(h^{3})
    \\
    &=
    y'_n + \frac{h}{2}y''_n + \frac{h^{2}}{6}y'''_n
    - f(t_n, y_n) - \frac{h}{2}Df(t_n,y_n)((1,y'_n))+
    O(h^{3})
    \\
    &=
    \frac{h^{2}}{6}y'''_n + O(h^{3})
  .\end{align}
  Luego, el método es consistente de orden $2$ y el error de
  truncamiento está acotado como
  \begin{equation}
    \tau(h)
    \leq \frac{h^{2}}{6}M + O(h^{3})
  ,\end{equation}
  donde $M$ es una constante con $|y'''(t)|\leq M$ para todo $t$ en el
  intervalo de tiempo considerado.
  Nótese que el tèrmino $O(h^{3})$ dependerá de las derivadas de orden
  superior de $y$ y de $f$.
  Para el problema $y'=-2t$, tenemos $f(t,y)=-2t$, así que
  \begin{align}
    y'''(t)
    &= \frac{d}{dt}Df(t,y(t))((1,y'(t))) \\
    &= \frac{d}{dt}Df(t,y(t))((1,-2t)) \\
    &= \frac{d}{dt}(-2,0)\cdot (1,-2t) \\
    &= \frac{d}{dt}(-2) \\
    &= 0
  .\end{align}
  Por lo tanto, todas las derivadas superiores de $y$ también son
  cero. Similarmente, las derivadas $D^{n}f$ son cero para $n\geq 2$.
  así que, de hecho, para este problema particular el término de error
  $O(h^{3})$ se anula. Dado que el término principal también se anula
  (pues está acotado por $y'''=0$) se sigue que
  $\tau_{n+1}(h)=0$ para todo $n$ y todo $h$, es decir, no hay error
  de truncamiento local, así que el método produce la solución
  exacta.
\end{sol}


\end{document}
