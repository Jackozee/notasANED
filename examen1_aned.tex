\documentclass[11pt,letterpaper]{article}
\usepackage[left=3.5cm,right=3.5cm,top=2.5cm,bottom=2.5cm]{geometry}
%\usepackage[spanish]{babel}

\usepackage{amsmath,amsfonts,amsthm}
\usepackage{enumitem,mathtools}
\setenumerate[0]{label=(\alph*)}

\newtheorem{exercise}{Ejercicio}
\newtheorem*{sol}{Solución}

\newcommand\N{\mathbb N}

\title{Primer examen parcial - Análisis numérico para ecuaciones
diferenciales}
\author{Jorge Alfredo Álvarez Contreras}

\begin{document}
\maketitle

\begin{exercise}
  Considere el problema de valores iniciales
  \begin{equation}
    y' = \frac{y+t}{y-t}, \quad y(0)=1
  ,\end{equation}
  cuya solución es $y(t)=t+\sqrt{1+2t^{2}}$. Utilice $h=0.1$ para
  \begin{enumerate}
    \item
      realizar dos pasos del método de Euler explícito
    \item
      realizar dos pasos del método de Euler implícito (utilice
      iteración Newton-Raphson con una tolerancia de $10^{-5}$).
  \end{enumerate}
\end{exercise}
\begin{sol}
  \begin{enumerate}
    \item
      Para $h=0.1$, tenemos $t_0=0$, $t_1=h=0.1$, así que el método de
      Euler explícito
      \begin{equation}
        u_{n+1} = u_n + hf_n
      \end{equation}
      comienza con $u_0=y(0)=1$ y produce las aproximaciones
      \begin{align}
        u_1
        &= u_0+hf_0 \\
        &= u_0+h \frac{u_0+t_0}{u_0-t_0} \\
        &= 1+h \\
        &= 1.1,
        \\
        u_2
        &= u_1+hf_1 \\
        &= u_1+h \frac{u_1+t_1}{u_1-t_1} \\
        &= 1.1+h \frac{1.1+0.1}{1.1-0.1} \\
        &= 1.1+(0.1)(1.2) \\
        &= 1.1 + 0.12 \\
        &= 1.22
      .\end{align}
    \item
      Para el método de Euler implícito,
      \begin{equation}
        u_{n+1} = u_n + hf_{n+1}
      ,\end{equation}
      también tenemos $t_0=0$ y
      $t_1=h=0.1$ y comenzamos con $u_0=y(0)=1$.
      Así, debemos resolver las ecuaciones
      \begin{equation}
        \left\{
          \begin{aligned}
            u_1 &= u_0 + hf_{1},
            \\
            u_2 &= u_1 + hf_{2}.
          \end{aligned}
        \right.
      \end{equation}
      Esto es
      \begin{equation}
        \left\{
          \begin{aligned}
            u_1 &= u_0 + h \frac{u_1+t_1}{u_1-t_1}
            \\
            u_2 &= u_1 + h \frac{u_2+t_2}{u_2-t_2}
          \end{aligned}
        \right.
      \end{equation}
      
      
  \end{enumerate}
\end{sol}


\end{document}
