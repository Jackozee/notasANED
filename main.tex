\documentclass[11pt,letterpaper]{article}
\usepackage[left=3.5cm,right=3.5cm,top=2.5cm,bottom=2.5cm]{geometry}
%\usepackage[spanish]{babel}

\usepackage{amsmath,amsfonts,amsthm}
\usepackage{enumitem,mathtools}
\usepackage[english]{isodate}
\setenumerate[0]{label=(\alph*)}

\newtheorem{exercise}{Ejercicio}
\newtheorem{example}{Ejemplo}
\newtheorem{remark}{Observación}
\newtheorem*{sol}{Solución}

\newcommand\N{\mathbb N}

\title{Notas de análisis numérico para ecuaciones diferenciales}
\author{Jorge Alfredo Álvarez Contreras}

\begin{document}
\maketitle

\section{Métodos de un paso}

\subsection{Estabilidad absoluta}

\begin{example}
  Consideremos el PVI
  \begin{equation}
    \left\{
    \begin{aligned}
      y' &= -2y \\
      y(0) &= 1
    \end{aligned}
    \right.
  \end{equation}
  La solución es, claramente, $y(t)=e^{-2t}$.

  El método de Euler es
  \begin{equation}
    \left\{
    \begin{aligned}
      u_{n+1} &= u_n + hf(t_n,u_n) \\
      u_0 &= 1
    \end{aligned}
    \right.
  .\end{equation}
  Es decir, la función de incremento es simplemente $\Phi=f$.
  Como $\Phi(t,y)=f(t,y)=-2y$ es Lipschitz en su segundo argumento
  podemos concluir que el método es consistente (de orden $1$).
  También es convergente.
  \begin{equation}
    \left\{
    \begin{aligned}
      u_{n+1} &= u_n + -2u_nh = (1-2h)u_n \\
      u_0 &= 1
    \end{aligned}
    \right.
  .\end{equation}
  

\end{example}

\begin{remark}
  No todos los métodos implicitos son inconcidionalmente estables.
\end{remark}

\subsection{Ecuaciones en diferencias }

Consideremos ecuaciones en diferencias de la forma
\begin{equation}\label{eq:ec_diferencias}
  u_{n+k} + \alpha_{k-1}u_{n+k-1}
  + \dots +
  \alpha_1 u_{n+1}+\alpha_0u_n
  =
  \phi_{n+k},
  \quad \alpha_0 \neq 0; n=0,1,\dots
\end{equation}
(dados $u_0,u_1,\dots,u_{k-1}$ ).
Si $\phi_{n+k}=0,n=0,1,\dots$, decimos que la ecuación en diferencias
es homogénea. Si $\alpha_{k-1},\alpha_{k-2},\dots,\alpha_0$ son
constantes, decimos que es una ecuación en diferencias con
coeficientes constantes.

Dados los valores iniciales $u_0,u_1,\dots,u_{k-1}$, se puede
determinar $u_n$ directamente a partir de la ecuación en diferencias,
sustituyendo sucesivamente.

Consideremos el problema homogéneo y busquemos soluciones de la forma
$u_n=r^{n}$. Entonces
\begin{equation}
  r^{n+k} + \alpha_{k-1}r^{n+k-1}
  + \dots +
  \alpha_1 r^{n+1}+\alpha_0r^n
  =
  0.
\end{equation}
Para $r=0$, obtenemos la solución trivial, así que supondremos que
$r\neq 0$, de modo que al cancelar $r^{n}$ obtenemos
\begin{equation}
  r^{k} + \alpha_{k-1}r^{k-1}
  + \dots +
  \alpha_1 r^{1}+\alpha_0
  =
  0.
\end{equation}
El lado izquierdo de esta ecuación se es un polinomio de grado $k$ y
se llama el polinomio característico de la ecuación en diferencias
\eqref{eq:ec_diferencias} 
Se denota con $\pi(r)$.
Si $\pi(r)$ tiene $k$ raíces distintas, digamos
$r_0,r_1,\dots,r_{k-1}$, entonces la solución general de la ecuación
en diferencias es
\begin{equation}\label{eq:sol_general}
  u_n = \gamma_0r_0^{n} + \dots + \gamma_{k-1}r_{k-1}^{n}
\end{equation}
y, dados $u_0,\dots,u_{k-1}$, existe un conjunto único de constantes
$\gamma_0,\dots,\gamma_{k_1}$ tal que \eqref{eq:sol_general}.

\begin{example}
  La sucesión de Fibonacci es la solución a la ecuación
  \begin{equation}
    u_{n+2} = u_{n+1} + u_n
  \end{equation}
  con condiciones iniciales $u_0=u_1=1$.
  Las soluciones de la forma $u_n=r^n$ satisfacen la ecuación
  característica
  \begin{equation}
    r^2 = r + 1
  .\end{equation}
  Es decir,
  \begin{equation}
    r = \frac{1\pm\sqrt 5}{2}
  .\end{equation}
  Por lo tanto,
  \begin{equation}
    u_n = \gamma_0
    \left( \frac{1+\sqrt 5}{2} \right)^n
    +
    \gamma_1
    \left( \frac{1-\sqrt 5}{2} \right)^n
  ,\end{equation}
  donde las constantes $\gamma_0$ y $\gamma_1$ se obtienen aplicando
  las condiciones iniciales
  \begin{align}
    1 = u_0 &= \gamma_0
    +
    \gamma_1
    \\
    1 = u_1 &= \gamma_0
    \left( \frac{1+\sqrt 5}{2} \right)
    +
    \gamma_1
    \left( \frac{1-\sqrt 5}{2} \right)
  ,\end{align}
  Resolviendo este sistema, obtenemos
  \begin{equation}
    \gamma_{0,1} = \frac{1}{2} \pm \frac{\sqrt 5}{10}
  .\end{equation}
\end{example}

Si la raíz $r_j$ tiene multiplicidad $m\geq 2$, debemos buscar
soluciones de la forma $p(n)r^n$, donde $p$ es un polinomio en $n$ de
grado $m-1$.

\begin{example}
  Consideremos la ecuación
  \begin{equation}
    u_{n+3} -2u_{n+2} - 7 u_{n+1} - 4u_n = 0
  \end{equation}
  con condiciones iniciales $u_0=1$, $u_1=0$, $u_{2}=-1$.
  La ecuación característica
  \begin{equation}
    r^{3}-2r^{2}-7r-4 = 0
  \end{equation}
  tiene una raíz simple $r_0=4$ y una raíz doble $r_{1,2}=-1$.
  Por lo tanto, la solución general es
  \begin{equation}
    u_n = \gamma_{0}4^n + (\gamma_1+\gamma_2n)(-1)^n
  .\end{equation}
  Aplicando las condiciones iniciales, obtenemos ecuaciones
  \begin{align}
    \gamma_0+\gamma_1 &= 1 \\
    4\gamma_0-\gamma_1-\gamma_2 &= 0 \\
    16\gamma_0+\gamma_1+2\gamma_2&= -1
  \end{align}
  con solución $\gamma_0=0$, $\gamma_1=1$, $\gamma_2=-1$.
  Por lo tanto, la solución es
  \begin{equation}
    u_n = (1-n)(-1)^{n}
  .\end{equation}
\end{example}

\section{Métodos multipaso}


\end{document}
