\documentclass[11pt]{article}
\usepackage[left=3.5cm,right=3.5cm,top=2.5cm,bottom=2.5cm]{geometry}
%\usepackage[spanish]{babel}

\usepackage{amsmath,amsfonts,amsthm}
\usepackage{enumitem,mathtools,graphicx}
\setenumerate[0]{label=(\alph*)}

\newtheorem{definition}{Definición}
\newtheorem{exercise}{Ejercicio}
\newtheorem*{sol}{Solución}
\newtheorem*{theorem}{Teorema}

\newcommand\N{\mathbb N}
\newcommand\R{\mathbb R}
\newcommand\C{\mathbb C}

\title{Análisis numérico para ecuaciones diferenciales \\
Tarea 2 - Estabilidad absoluta, ecuaciones en diferencias finitas y
métodos multipaso}
\author{Jorge Alfredo Álvarez Contreras}

\begin{document}
\maketitle

\begin{exercise}
   Se sabe que al aplicar el método de Euler al problema de valores
   iniciales $y'=y$, $y(0)=1$, se obtienen las aproximaciones
   $u_n=(1+h)^{t_n / h}$.
   \begin{enumerate}
     \item
       Demuestre que $y_n-u_n = \frac{1}{2}ht_ne^{t_n}+O(h^{2})$.
     \item
       Usando el resultado anterior, determine el tamaño de paso más
       grande posible para aproximar $y(1)$ con una tolerancia de
       $10^{-5}$. ¿Cuántos pasos de Euler se tienen que dar para
       obtener dicha aproximación de $y(1)$?
   \end{enumerate}
\end{exercise}
\begin{sol}
  Dado que $u_n=(1+h)^{t_n/h}$, tenemos
  \begin{align}
    \ln u_n 
    &= \frac{t_n}{h}\ln(1+h) \\
    &= \frac{t_n}{h}\left(h-\frac{h^{2}}{2} + O(h^{3})\right) \\
    &= t_n - t_n \frac{h}{2} + O(h^{2})
  .\end{align}
  Esto significa que tenemos una función $f(h)$, un $h_0>0$ y una
  constante $C>0$ tales que, para todo $0<h<h_0$, se satisfacen
  $|f(h)|\leq Ch^{2}$ y
  \begin{equation}
    \ln u_n = t_n - t_n \frac{h}{2} + f(h)
  .\end{equation}
  Notemos que, para $0<h<h_0$, tenemos
  \begin{align}
     \left| -t_n \frac{h}{2} + f(h) \right|^{2}
     &\leq t_n^{2} \frac{h^{2}}{4} + t_n h |f(h)| + |f(h)|^{2} \\
     &\leq T^{2} \frac{h^{2}}{4} + t_n h Ch^{2} + C^{2}h^{4} \\
     &\leq \left(\frac{T^{2}}{4} + t_n h_0 C + C^{2}h_0 \right)h^{2}
  .\end{align}
  Por lo tanto,
  \begin{equation}
     \left( -t_n \frac{h}{2} + f(h) \right)^{2}
     = O(h^{2})
  .\end{equation}
  
   
  Luego,
  \begin{align}
    u_n
    &= e^{t_n}e^{-t_n \frac{h}{2} + f(h)} \\
    &= e^{t_n}
        \left(1
         -t_n \frac{h}{2} + f(h)
         + O\left(\left(
         -t_n \frac{h}{2} + f(h)
         \right)^{2}
         \right)
      \right) \\
    &= 
        e^{t_n}
         -t_n \frac{h}{2}e^{t_n} + O(h^{2})
  .\end{align}
  Ahora, notando que la solución analítica es $y(t)=e^{t}$, tenemos
  $y_n=e^{t_n}$, por lo cual
  \begin{align}
    y_n - u_n
    &= e^{t_n} - u_n \\
    &= t_n \frac{h}{2} e^{t_n} + O(h^{2})
  ,\end{align}
  como se quería.
\end{sol}

\begin{exercise}
  Determine y grafique (con ayuda de \texttt{python}) la región de
  estabilidad absoluta de los métodos
  \begin{enumerate}
    \item
      Heun
    \item
      BDF2
  \end{enumerate}
\end{exercise}
\begin{sol}
  \begin{enumerate}
    \item
      El método de Heun es
      \begin{equation}
        u_{n+1} = u_n + \frac{h}{2}[f_n + f(t_{n+1},u_n+hf_n)]
      .\end{equation}
      Para el problema de prueba $y'=y$, tenemos $f(t,y)=\lambda y$, así
      que
      \begin{align}
        f_n
          &= f(t_n,u_n)
          = \lambda u_n \\
        f(t_{n+1},u_{n}+hf_n)
          &= \lambda(u_{n}+hf_n)
          = \lambda u_n + h\lambda^{2} u_n
      .\end{align}
      Luego, el método es
      \begin{align}
        u_{n+1}
        &= u_n + \frac{h}{2}(2\lambda u_n + h\lambda^{2} u_n) \\
        &= \left(1 + h\lambda + \frac{1}{2}(h\lambda)^{2}\right) u_n
      \end{align}
      con $u_0=1$.
      Por lo tanto, la solución es
      \begin{equation}
        u_n = \left(1 + h\lambda + \frac{1}{2}(h\lambda)^{2}\right)^{n}
      .\end{equation}
      y la región de estabilidad absoluta del método consiste en los
      $z\in\C$ tales que
      \begin{equation}
        \left(1 + z + \frac{1}{2}z^{2}\right)^{n} \to 0
      .\end{equation}
      Esto sucede si, y solo si, 
      \begin{equation}
        \left|1 + z + \frac{1}{2}z^{2}\right|^{n} \to 0
      .\end{equation}
      lo cual sucede exactamente cuando
      \begin{equation}
        \left|1 + z + \frac{1}{2}z^{2}\right|^{2}<1
      .\end{equation}
    %  \begin{align}
    %    \left|1 + z + \frac{1}{2}z^{2}\right|^{2}
    %    &=
    %    \left(1 + z + \frac{1}{2}z^{2}\right)
    %    \left(1 + \bar z + \frac{1}{2}\bar z^{2}\right)
    %    \\
    %    &=
    %    1 + \bar z + \frac{1}{2}\bar z^{2}
    %    + z + z\bar z + \frac{1}{2}z\bar z^{2}
    %    + \frac{1}{2}z^{2} + \frac{1}{2}z^{2}\bar z + \frac{1}{4}z^{2}\bar z^{2}
    %    \\
    %    &=
    %    1 + 2\Re(z) + \Re(z^{2})
    %    + |z|^{2} + |z|^{2}\Re(z)
    %    + \frac{1}{4}|z|^{4}
    %    \\
    %    &=
    %    1 + 2x + x^{2}-y^{2}
    %    + x^{2}+y^{2} + (x^{2}+y^{2})x
    %    + \frac{1}{4}(x^{2}+y^{2})^{2}
    %    \\
    %    &=
    %    1 + 2x + 2x^{2}
    %    + (x^{2}+y^{2})x
    %    + \frac{1}{4}(x^{2}+y^{2})^{2}
    %  .\end{align}
      
    %  Si $x,y$ son las partes real e imaginaria de $z$, entonces esto es
    %  \begin{equation}
    %    (1+x+iy + (x^{2}-y^{2}) / 2 + ixy)
    %    (1+x-iy + (x^{2}-y^{2}) / 2 - ixy) < 1
    %  .\end{equation}
    \item
      El método BDF2 es
      \begin{equation}
        \frac{1}{2h}(u_{n-1} - 4 u_n + 3 u_{n+1}) - f_{n+1} = 0
      .\end{equation}
      Para el problema de prueba $f(t,y)=\lambda y$, tenemos
      \begin{equation}
        u_{n-1} - 4 u_n + 3 u_{n+1} - 2h\lambda u_{n+1} = 0
      .\end{equation}
      Así, el polinomio característico es
      \begin{equation}
        \pi(r;h\lambda) = \rho(r) - h\lambda \sigma(r)
      \end{equation}
      donde
      \begin{align}
        \rho(r) &= 1 - 4 r + 3 r^{2} \\
        \sigma(r) &= 2r^{2}
      .\end{align}
      Por lo tanto,
      
  \end{enumerate}
    
\end{sol}

\begin{exercise}
  Resuelva la ecuación en diferencias
  \begin{equation}
    u_{n+4} - 6u_{n+3} + 14u_{n+2} - 16 u_{n+1} + 8u_{n} = n
  \end{equation}
  con las condiciones $u_0=1$, $u_1=2$, $u_2=3$, $u_3=4$.

  \emph{Solución: $u_n=2^n(n / 4 -1) + 2^{(n-2) / 2}\sin(n\pi /
  4)+n+2$}.
\end{exercise}

\begin{exercise}
  Calcule el error de truncamiento local $\tau_{n+1}(h)$ y la
  constante de error del método de Adams-Moulton AM3.
\end{exercise}


\end{document}
