\documentclass[11pt]{article}
\usepackage[left=3.5cm,right=3.5cm,top=2.5cm,bottom=2.5cm]{geometry}
%\usepackage[spanish]{babel}
\overfullrule = 5mm

\usepackage{amsmath,amsfonts,amsthm}
\usepackage{enumitem,mathtools,graphicx}
\setenumerate[0]{label=(\alph*)}

\newtheorem{definition}{Definición}
\newtheorem{reminder}{Recordatorio}
\newtheorem{exercise}{Ejercicio}
\newtheorem*{sol}{Solución}
\newtheorem*{theorem}{Teorema}

\newcommand\fd{\mathrm{fd}}
\newcommand\N{\mathbb N}
\newcommand\R{\mathbb R}
\newcommand\C{\mathbb C}
\newcommand\ol\overline

%\usepackage{csquotes}
%\usepackage[style=authoryear]{biblatex}
%\addbibresource{references.bib}

\title{Análisis numérico para ecuaciones diferenciales \\
Tarea 4 - Diferencias finitas}
\author{Jorge Alfredo Álvarez Contreras}

\begin{document}
\maketitle

\begin{reminder}
\end{reminder}

\section*{Ejercicios}

\begin{exercise}
  Considere el problema de valores en la frontera
  \begin{equation}
    \left\{
      \begin{aligned}
        u''(x) - p(x)u'(x) - q(x)u(x) &= f(x), \quad x\in(a,b)
        \\
        u(a) = \alpha, \quad u(b)=\beta, &
      \end{aligned}
    \right.
  \end{equation}
  donde $q(x)\geq q_0>0$ para $x\in[a,b]$. Desarrolle una
  discretización en diferencias finitas usando diferencias centradas
  para aproximar $u''$ y $u'$.
  \begin{enumerate}
    \item
      Estime el error de truncamiento local.
    \item
      Demuestre que el problema discreto tiene solución para $h$ 
      suficientemente pequeño.
    \item
      Utilice el método desarrollado en este ejercicio para resolver
      el problema
      \begin{equation}
        \left\{
          \begin{aligned}
            -u'' + \alpha u' + u &= 0,
            \\
            u(0) = 0,\quad u(1)=1,
          \end{aligned}
        \right.
      \end{equation}
      para diferentes valores de la constante $\alpha>0$. Muestre sus
      resultados de manera gráfica (solución v.s. aproximación y error
      puntual).
  \end{enumerate}
\end{exercise}
\begin{sol}
  Discretizamos el espacio en $n+2$ puntos $x_0,\dots,x_{n+1}$. Para
  los puntos interiores, las diferencias centradas son
  \begin{align}
    u(x_i)
    &\to
    u_i
    \\
    u'(x_i)
    &\to
    \frac{u_{i+1}-u_{i-1}}{2h}
    \\
    u''(x_i)
    &\to
    \frac{u_{i+1}-2u_{i}+u_{i-1}}{h^{2}}
  .\end{align}
  Sustituyendo en el problema original, obtenemos
  el problema discretizado
  \begin{equation}
    \frac{u_{i+1}-2u_{i}+u_{i-1}}{h^{2}}
    - p_i
    \frac{u_{i+1}-u_{i-1}}{2h}
    - q_i
    u_i
    =
    f_i,
    \quad i=1,\dots,n
  ,\end{equation}
  donde $x_0=a$, $x_{n+1}=b$, $p_i=p(x_i)$, $q_i=q(x_i)$ y
  $f_i=f(x_i)$. Así, obtenemos un sistema de ecuaciones
  \begin{align}
      \frac{1}{h^{2}}(u_{2}-2u_{1}+u_{0})
      - 
      \frac{p_1}{2h}(u_{2}-u_{0})
      - q_1
      u_1
      &=
      f_1,
    \\
      \frac{1}{h^{2}}(u_{3}-2u_{2}+u_{1})
      - 
      \frac{p_2}{2h}(u_{3}-u_{1})
      - q_2
      u_2
      &=
      f_2,
    \\
      &\vdots \nonumber
    \\
      \frac{1}{h^{2}}(u_{n}-2u_{n-1}+u_{n-2})
      - 
      \frac{p_{n-1}}{2h}(u_{n}-u_{n-2})
      - q_{n-1}
      u_{n-1}
      &=
      f_{n-1},
    \\
      \frac{1}{h^{2}}(u_{n+1}-2u_{n}+u_{n-1})
      - 
      \frac{p_n}{2h}(u_{n+1}-u_{n-1})
      - q_n
      u_n
      &=
      f_n.
  \end{align}
  Aplicando las condiciones de frontera $u_0=\alpha$, $u_{n+1}=\beta$,
  obtenemos
  \begin{align}
      - \frac{1}{h^{2}}(-u_{2}+2u_{1})
      - \frac{p_1}{2h}u_{2}
      - q_1 u_1
      &=
      f_1 - \frac{1}{h}\left(\frac{1}{h} + \frac{p_1}{2}\right)\alpha,
    \\
      - \frac{1}{h^{2}}(-u_{3}+2u_{2}-u_{1})
      - \frac{p_2}{2h}(u_{3}-u_{1})
      - q_2 u_2
      &=
      f_2,
    \\
      &\vdots \nonumber
    \\
      - \frac{1}{h^{2}}(-u_{n}+2u_{n-1}-u_{n-2})
      - \frac{p_{n-1}}{2h}(u_{n}-u_{n-2})
      - q_{n-1} u_{n-1}
      &=
      f_{n-1},
    \\
      - \frac{1}{h^{2}}(2u_{n}-u_{n-1})
      + \frac{p_n}{2h}u_{n-1}
      - q_n u_n
      &=
      f_n - \frac{1}{h}\left(\frac{1}{h} - \frac{p_n}{2}\right)\beta.
  \end{align}
  que se puede resumir en la ecuación matricial
  \begin{equation}
    -\frac{1}{h^{2}}AU - \frac{1}{2h}PU - QU = F - \frac{1}{h}E
  \end{equation}
  o, equivalentemente,
  \begin{equation}\label{eq:metodo_matr}
    \left(A + \frac{1}{2}hP + h^{2}Q\right)U = -h^{2}F + hE
  ,\end{equation}
  donde $E$ y $F$ son los vectores
  \begin{equation}
      F =
      \begin{bmatrix}
        f_1 \\ \vdots \\ f_n
      \end{bmatrix},
      \qquad\qquad
      E =
      \begin{bmatrix}
        \left(\frac{1}{h} + \frac{p_1}{2}\right)\alpha
        \\
        0 \\
        \vdots \\
        0 \\
        \left(\frac{1}{h} - \frac{p_n}{2}\right)\beta
      \end{bmatrix}
  \end{equation}
  y $A,P,Q$ son las matrices
  \begin{align}
      A
      &=
      \begin{bmatrix}
        2 & -1 \\
        -1 & 2 & -1 \\[-2mm]
           & -1 & 2 & \ddots \\
           & & \ddots & \ddots & \ddots \\
           &  & & \ddots & 2 & -1 \\
           &&&& -1 & 2 & -1 \\
           &&&& & -1 & 2 \\
      \end{bmatrix}
    \\
      P
      &=
      \begin{bmatrix}
        0 & p_1 \\
        -p_2 & 0 & p_2 \\[-2mm]
           & -p_3 & 0 & \ddots \\
           & & \ddots & \ddots & \ddots \\
           &  & & \ddots & 0 & p_{n-2} \\
           &&&& -p_{n-1} & 0 & p_{n-1} \\
           &&&& & -p_n & 0
      \end{bmatrix}
    \\
      Q
      &=
      \begin{bmatrix}
        q_1 \\
        & q_2 \\
        & & \ddots \\
        & & & q_{n-1} \\
        & & & & q_{n}
      \end{bmatrix}
  .\end{align}
  \begin{enumerate}
    \item
      El método obtenido está dado por la solución a la ecuación en
      diferencias
      \begin{equation}
        \frac{u_{i+1}-2u_{i}+u_{i-1}}{h^{2}}
        - p_i
        \frac{u_{i+1}-u_{i-1}}{2h}
        - q_i
        u_i
        =
        f_i,
        \quad i=1,\dots,n
      .\end{equation}
      Denotemos el lado izquierdo por $L_hu_i$, de modo que
      \begin{equation}
        L_hu_i = f_i,
        \quad i=1,\dots,n
      .\end{equation}
      Entonces el error de truncamiento local
      $\tau_{i}(h)$ está definido por
      \begin{equation}
        L_hu(x_i) = f_i + \tau_{i}(h)
      ,\end{equation}
      donde $u(x)$ es la solución exacta al problema continuo.
      Recordemos que, si tenemos cotas $\|u'''\|\leq M$ y
      $\|u''''\|\leq N$, entonces los esquemas de diferencias
      centradas satisfacen
      \begin{align}
        \frac{u(x_{i+1})-u(x_{i-1})}{2h}
        &= 
        u'(x_i)
        + \frac{h^{2}}{6}M
        + O(h^{3})
        \\
        \frac{u(x_{i+1})-2u(x_{i})+u(x_{i-1})}{h^{2}}
        &=
        u''(x_i)
        + \frac{h^{2}}{12}N
        + O(h^{3})
      .\end{align}
      Por lo tanto,
      \begin{align}
        L_hu(x_i)
          &=
          \frac{u(x_{i+1})-2u(x_{i})+u(x_{i-1})}{h^{2}}
          - p_i
          \frac{u(x_{i+1})-u(x_{i-1})}{2h}
          - q_i
          u(x_i)
        \\
        &=
          \left(
            u''(x_i)
            + \frac{h^{2}}{12}N
            + O(h^{3})
          \right)
        - p_i
          \left(
            u'(x_i)
            + \frac{h^{2}}{6}M
            + O(h^{3})
          \right)
        - q_i
        u(x_i)
        \\
        &=
          \left( u''(x_i) - p_i u'(x_i) - q_i u(x_i) \right)
          + \frac{h^{2}}{12}N
          - p_i \frac{h^{2}}{6}M
            + O(h^{3})
        \\
        &=
          f_i
          +
          \left( \frac{N}{2} - p_i M \right)
          \frac{h^{2}}{6}
            + O(h^{3})
      .\end{align}
      Así,
      \begin{equation}
        \tau_i(h)
        =
        \left( \frac{N}{2} - p_i M \right)
        \frac{h^{2}}{6}
        + O(h^{3})
      ,\end{equation}
      con lo cual el método es consistente de orden $2$.
    \item
      El problema tiene solución si podemos resolver para $U$ de la
      ecuación
      \begin{equation}
        \left(A + \frac{1}{2}hP + h^{2}Q\right)U = -h^{2}F + hE
      ,\end{equation}
      donde $A,P,Q,F$ y $E$ están dadas arriba.
      Para esto, basta ver que la matríz entre paréntesis es
      invertible para $h$ suficientemente pequeño.
      Para esto, consideremos la función
      \begin{equation}
        \phi(h) = \det \left(A + \frac{1}{2}hP + h^{2}Q\right)
      .\end{equation}
      Notemos que $\phi(0)=\det A >0$, pues $A$ es definida positiva.
      Como el determinante de una matríz $n$ por $n$ es una función
      polinomial en sus $n^{2}$ entradas, entonces $\phi$ es una
      función continua de $h$ (aquí estamos considerando que $h$ puede
      variar aún cuando ya tenemos una $n$ fija).
      Por lo tanto, para $\epsilon=(\det A) / 2>0$, podemos obtener
      $\delta>0$ tal que toda $h$ con $|h|<\delta$ satisface
      \begin{equation}
        |\phi(h)-\phi(0)|<(\det A) / 2
      ,\end{equation}
      por lo cual
      \begin{equation}
        0 < \det A = \phi(0) = |-\phi(0)| = 
        |\phi(h) - \phi(0) - \phi(h)|
        \leq (\det A) / 2 + |\phi(h)|
      .\end{equation}
      Luego, si $|h|<\delta$, entonces
      \begin{equation}
        |\phi(h)| \geq (\det A) / 2 > 0
      ,\end{equation}
      de donde se sigue que
      \begin{equation}
        \phi(h) = \det \left(A + \frac{1}{2}hP + h^{2}Q\right) \neq 0
      ,\end{equation}
      así que $A + \frac{1}{2}hP + h^{2}Q$ es invertible y el problema
      tiene solución
       \begin{equation}
        U = \left(A + \frac{1}{2}hP + h^{2}Q\right)^{-1}
        (-h^{2}F + hE)
      .\end{equation}
    \item
      Para el problema
      \begin{equation}
        \left\{
          \begin{aligned}
            -u'' + \alpha u' + u &= 0,
            \\
            u(0) = 0,\quad u(1)=1,
          \end{aligned}
        \right.
      \end{equation}
      tenemos $p_i=\alpha$, $q_i=1$, $f_i=0$, $\alpha=0$ y $\beta=1$,
      así que
      \begin{align}
          A
          &=
          \begin{bmatrix}
            2 & -1 \\
            -1 & 2 & -1 \\[-2mm]
               & -1 & 2 & \ddots \\
               & & \ddots & \ddots & \ddots \\
               &  & & \ddots & 2 & -1 \\
               &&&& -1 & 2 & -1 \\
               &&&& & -1 & 2 \\
          \end{bmatrix}
        \\
          P
          &=
          \alpha
          \begin{bmatrix}
            0 & 1 \\
            -1 & 0 & 1 \\[-2mm]
               & -1 & 0 & \ddots \\
               & & \ddots & \ddots & \ddots \\
               &  & & \ddots & 0 & 1 \\
               &&&& -1 & 0 & 1 \\
               &&&& & -1 & 0
          \end{bmatrix}
        \\
          Q
          &= I
      .\end{align}
      
      
      
      
  \end{enumerate}

\end{sol}

\begin{exercise}
  Implemente en Python la solución numérica con diferencias finitas
  del siguiente problema de valores en la frontera
  \begin{equation}
    \left\{
      \begin{aligned}
        -\Delta u &= 2\pi^{2}\sin(\phi x)\sin(\pi y)
        && \text{ en } \Omega = (0,1) ^{2},
        \\
        u &= 0 && \text{ en } \partial\Omega
      \end{aligned}
    \right.
  \end{equation}
  para $h=0.25,0.1,0.05$ y grafique $u_h,u-u_h$ y calcule
  $\|u-u_h\|_{h,\infty}$ tomando en cuenta que la solución es
  $u(x,y)=\sin(\pi x)\sin(\pi y)$.
\end{exercise}

\begin{exercise}
  Demuestre que la matríz $A_{\fd}$ obtenida de discretizar la EDP de
  Poisson es simétrica y definida positiva para el caso 1D y 2D.
\end{exercise}

\end{document}
