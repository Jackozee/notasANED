\documentclass[11pt]{article}
\usepackage[left=3.5cm,right=3.5cm,top=2.5cm,bottom=2.5cm]{geometry}
%\usepackage[spanish]{babel}
\overfullrule = 5mm

\usepackage{amsmath,amsfonts,amsthm}
\usepackage{enumitem,mathtools,graphicx}
\setenumerate[0]{label=(\alph*)}

\newtheorem{definition}{Definición}
\newtheorem{reminder}{Recordatorio}
\newtheorem{exercise}{Ejercicio}
\newtheorem*{sol}{Solución}
\newtheorem*{theorem}{Teorema}

\newcommand\fd{\mathrm{fd}}
\newcommand\N{\mathbb N}
\newcommand\R{\mathbb R}
\newcommand\C{\mathbb C}
\newcommand\ol\overline

%\usepackage{csquotes}
%\usepackage[style=authoryear]{biblatex}
%\addbibresource{references.bib}

\title{Análisis numérico para ecuaciones diferenciales \\
Tarea 4 - Diferencias finitas}
\author{Jorge Alfredo Álvarez Contreras}

\begin{document}
\maketitle

\begin{reminder}
\end{reminder}

\section*{Ejercicios}

\begin{exercise}
  Considere el problema de valores en la frontera
  \begin{equation}
    \left\{
      \begin{aligned}
        u''(x) - p(x)u'(x) - q(x)u(x) &= f(x), \quad x\in(a,b)
        \\
        u(a) = \alpha, \quad u(b)=\beta, &
      \end{aligned}
    \right.
  \end{equation}
  donde $q(x)\geq q_0>0$ para $x\in[a,b]$. Desarrolle una
  discretización en diferencias finitas usando diferencias centradas
  para aproximar $u''$ y $u'$.
  \begin{enumerate}
    \item
      Estime el error de truncamiento local.
    \item
      Demuestre que el problema discreto tiene solución para $h$ 
      suficientemente pequeño.
    \item
      Utilice el método desarrollado en este ejercicio para resolver
      el problema
      \begin{equation}
        \left\{
          \begin{aligned}
            -u'' + \alpha u' + u &= 0,
            \\
            u(0) = 0,\quad u(1)=1,
          \end{aligned}
        \right.
      \end{equation}
      para diferentes valores de la constante $\alpha>0$. Muestre sus
      resultados de manera gráfica (solución v.s. aproximación y error
      puntual).
  \end{enumerate}
\end{exercise}
\begin{sol}
  Supongamos que $u_i$ es la solución al problema discreto, que se
  obtiene haciendo las sustituciones
  \begin{align}
    u(x_i)
    &\to
    u_i
    \\
    u'(x_i)
    &\to
    \frac{u_{i+1}-u_{i-1}}{2h}
    \\
    u''(x_i)
    &\to
    \frac{u_{i+1}-2u_{i}+u_{i-1}}{h^{2}}
  .\end{align}
  Si $\|u'''\|\leq M$ y $\|u''''\|\leq N$, entonces tenemos
  \begin{align}
    u'(x_i)
    &= 
    \frac{u_{i+1}-u_{i-1}}{2h}
    - \frac{h^{2}}{6}M
    + O(h^{3})
    \\
    u''(x_i)
    &=
    \frac{u_{i+1}-2u_{i}+u_{i-1}}{h^{2}}
    - \frac{h^{2}}{12}N
    + O(h^{3})
  .\end{align}
  Entonces el problema discretizado es
  \begin{equation}
    \frac{u_{i+1}-2u_{i}+u_{i-1}}{h^{2}}
    - p_i
    \frac{u_{i+1}-u_{i-1}}{2h}
    - q_i
    u_i
    =
    f_i, \quad i=1,\dots,n
  ,\end{equation}
  donde $x_0=a$, $x_{n+1}=b$, $p_i=p(x_i)$ y $q_i=q(x_i)$.
  Entonces
  \begin{equation}
    (2-p_ih)u_{i+1}-(4-2h^{2}q_i)u_{i}+(2 + p_i h)u_{i-1}
    =
    2h^{2}f_i
  .\end{equation}
  Así
  obtenemos un sistema de ecuaciones
  \begin{align}
    (2-p_1h)u_{2}-(4-2h^{2}q_1)u_{1}+(2 + p_1 h)u_{0}
    &=
    2h^{2}f_1
    \\
    (2-p_2h)u_{3}-(4-2h^{2}q_2)u_{2}+(2 + p_2 h)u_{1}
    &=
    2h^{2}f_2
    \\
    &\vdots
    \\
    (2-p_{n-1}h)u_{n}-(4-2h^{2}q_{n-1})u_{n-1}+(2 + p_{n-1} h)u_{n-2}
    &=
    2h^{2}f_{n-1}
    \\
    (2-p_nh)u_{n+1}-(4-2h^{2}q_n)u_{n}+(2 + p_n h)u_{n-1}
    &=
    2h^{2}f_n
  \end{align}
  Aplicando las condiciones iniciales $u_0=\alpha$, $u_{n+1}=\beta$,
  obtenemos
  \begin{align}
    (2-p_1h)u_{2}-(4-2h^{2}q_1)u_{1}
    &=
    2h^{2}f_1
    -(2 + p_1 h)\alpha
    \\
    (2-p_2h)u_{3}-(4-2h^{2}q_2)u_{2}+(2 + p_2 h)u_{1}
    &=
    2h^{2}f_2
    \\
    &\vdots
    \\
    (2-p_{n-1}h)u_{n}-(4-2h^{2}q_{n-1})u_{n-1}+(2 + p_{n-1} h)u_{n-2}
    &=
    2h^{2}f_{n-1}
    \\
    -(4-2h^{2}q_n)u_{n}+(2 + p_n h)u_{n-1}
    &=
    2h^{2}f_n
    -(2-p_nh)\beta
  \end{align}
  que se puede resumir en la ecuación matricial
  \begin{equation}
    \begin{bmatrix}
      -(4-2h^{2}q_1) & (2-p_1h) \\
      (2+p_2h) & -(4-2h^{2}q_2) & (2-p_2h) \\
               & (2+p_3h) & -(4-2h^{2}q_3) & (2-p_3h) \\
               & & & & \ddots \\
               &&&&& (2+p_{n-1}h) & -(4-2h^{2}q_{n-1}) & (2-p_{n-1}h) \\
    \end{bmatrix}
  ,\end{equation}
  
  
  
\end{sol}

\begin{exercise}
  Implemente en Python la solución numérica con diferencias finitas
  del siguiente problema de valores en la frontera
  \begin{equation}
    \left\{
      \begin{aligned}
        -\nabla u &= 2\pi^{2}\sin(\phi x)\sin(\pi y),
        \quad \text{ en } \Omega = (0,1) ^{2},
        \\
        u &= 0 
        \quad \text{ en } \partial\Omega
      \end{aligned}
    \right.
  \end{equation}
  para $h=0.25,0.1,0.05$ y grafique $u_h,u-u_h$ y calcule
  $\|u-u_h\|_{h,\infty}$ tomando en cuenta que la solución es
  $u(x,y)=\sin(\pi x)\sin(\pi y)$.
\end{exercise}

\begin{exercise}
  Demuestre que la matríz $A_{\fd}$ obtenida de discretizar la EDP de
  Poisson es simétrica y definida positiva para el caso 1D y 2D.
\end{exercise}

\end{document}
