\documentclass[11pt]{article}
\usepackage[left=3.5cm,right=3.5cm,top=2.5cm,bottom=2.5cm]{geometry}
%\usepackage[spanish]{babel}
\overfullrule = 5mm

\usepackage{amsmath,amsfonts,amsthm}
\usepackage{enumitem,mathtools,graphicx}
\setenumerate[0]{label=(\alph*)}

\newtheorem{definition}{Definición}
\newtheorem{reminder}{Recordatorio}
\newtheorem{exercise}{Ejercicio}
\newtheorem*{sol}{Solución}
\newtheorem*{theorem}{Teorema}

\newcommand\fd{\mathrm{fd}}
\newcommand\N{\mathbb N}
\newcommand\R{\mathbb R}
\newcommand\C{\mathbb C}
\newcommand\ol\overline

%\usepackage{csquotes}
%\usepackage[style=authoryear]{biblatex}
%\addbibresource{references.bib}

\title{Análisis numérico para ecuaciones diferenciales \\
Tarea 4 - Diferencias finitas}
\author{Jorge Alfredo Álvarez Contreras}

\begin{document}
\maketitle

\begin{reminder}
\end{reminder}

\section*{Ejercicios}

\begin{exercise}
  Considere el problema de valores en la frontera
  \begin{equation}
    \left\{
      \begin{aligned}
        u''(x) - p(x)u'(x) - q(x)u(x) &= f(x), \quad x\in(a,b)
        \\
        u(a) = \alpha, \quad u(b)=\beta, &
      \end{aligned}
    \right.
  \end{equation}
  donde $q(x)\geq q_0>0$ para $x\in[a,b]$. Desarrolle una
  discretización en diferencias finitas usando diferencias centradas
  para aproximar $u''$ y $u'$.
  \begin{enumerate}
    \item
      Estime el error de truncamiento local.
    \item
      Demuestre que el problema discreto tiene solución para $h$ 
      suficientemente pequeño.
    \item
      Utilice el método desarrollado en este ejercicio para resolver
      el problema
      \begin{equation}
        \left\{
          \begin{aligned}
            -u'' + \alpha u' + u &= 0,
            \\
            u(0) = 0,\quad u(1)=1,
          \end{aligned}
        \right.
      \end{equation}
      para diferentes valores de la constante $\alpha>0$. Muestre sus
      resultados de manera gráfica (solución v.s. aproximación y error
      puntual).
  \end{enumerate}
\end{exercise}

\begin{exercise}
  Implemente en Python la solución numérica con diferencias finitas
  del siguiente problema de valores en la frontera
  \begin{equation}
    \left\{
      \begin{aligned}
        -\nabla u &= 2\pi^{2}\sin(\phi x)\sin(\pi y),
        \quad \text{ en } \Omega = (0,1) ^{2},
        \\
        u &= 0 
        \quad \text{ en } \partial\Omega
      \end{aligned}
    \right.
  \end{equation}
  para $h=0.25,0.1,0.05$ y grafique $u_h,u-u_h$ y calcule
  $\|u-u_h\|_{h,\infty}$ tomando en cuenta que la solución es
  $u(x,y)=\sin(\pi x)\sin(\pi y)$.
\end{exercise}

\begin{exercise}
  Demuestre que la matríz $A_{\fd}$ obtenida de discretizar la EDP de
  Poisson es simétrica y definida positiva para el caso 1D y 2D.
\end{exercise}

\end{document}
