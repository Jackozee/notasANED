\documentclass[11pt]{article}
\usepackage[left=3.5cm,right=3.5cm,top=2.5cm,bottom=2.5cm]{geometry}
%\usepackage[spanish]{babel}
\overfullrule = 5mm

\usepackage{amsmath,amsfonts,amsthm}
\usepackage{enumitem,mathtools,graphicx}
\setenumerate[0]{label=(\alph*)}

\newtheorem{definition}{Definición}
\newtheorem{exercise}{Ejercicio}
\newtheorem*{sol}{Solución}
\newtheorem*{theorem}{Teorema}

\newcommand\N{\mathbb N}
\newcommand\R{\mathbb R}
\newcommand\C{\mathbb C}
\newcommand\ol\overline

\usepackage{csquotes}
\usepackage[style=authoryear]{biblatex}
\addbibresource{references.bib}

\title{Análisis numérico para ecuaciones diferenciales \\
Tarea 3 - Métodos Runge-Kutta}
\author{Jorge Alfredo Álvarez Contreras}

\begin{document}
\maketitle

\begin{exercise}
   Programar el método Runge-Kutta Fehlberg para resolver el problema
   de valores iniciales
   \begin{equation}
     y'
     =
     (1-2x)y,
     \quad y(0)=1,
     \quad 0\leq x\leq 4
   ,\end{equation}
   tomando $h_\mathrm{max}=0.1$ y tolerancia $0.01$. Elaborar una
   gráfica del error y una gráfica de $x_n$ contra $h_n$. (Marque con
   un punto los pasos aceptados y con un asterisco los pasos
   rechazados).
\end{exercise}
\begin{sol}
\end{sol}

\begin{exercise}
  Considere la familia de métodos de Runge-Kutta correspondientes al
  table\-ro de Butcher
  \begin{equation}
    \begin{array}{c|cc}
      0 & 0 & 0 \\
      \frac{1}{2\theta} & \frac{1}{2\theta} & 0 \\[2mm]
      \hline \\[-3mm]
                        & 1-\theta & \theta.
    \end{array}
  \end{equation}
  Determine el orden del método y la función de estabilidad
  $R(h\lambda)$. Elabore una gráfica de la región de estabilidad
  absoluta.
\end{exercise}
\begin{sol}
\end{sol}

\begin{exercise}
  Considere el método de Runge-Kutta con tablero de Butcher dado por
  \begin{equation}
    \begin{array}{c|c}
      c & A \\
      \hline \\[-3mm]
        & b^T
    \end{array}
  \end{equation}
  donde $A$ es una matríz estrictamente triangular inferior. Demuestre
  que $R(h\lambda)$ es el polinomio en $z=h\lambda$ dado por 
   \begin{equation}
    R(z) = \det(I-zA+z\ol 1 b^T)
  .\end{equation}
\end{exercise}
\begin{sol}
  El método asociado al tablero de Butcher dado es
  \begin{equation}
    u_{n+1} = u_n + h F(t_n,u_n,h,f)
  \end{equation}
  donde
  \begin{align}
    F
      &=
      \sum_{i=1}^{s}b_ik_i
    \\[-3mm]
    k_i
      &= f(t_n+c_ih, u_n+h \sum_{j=1}^{s}a_{ij} k_j)
      \\[-4mm]
    \text{y} \quad
    c_i
      &= \sum_{j=1}^{s}a_{ij}
  .\end{align}
  Aplicado este método al problema de prueba $f(t,y)=\lambda y$,
  tenemos
  \begin{align}
    k_i
    &= \lambda(u_n+h \sum_{j=1}^{s}a_{ij} k_j) \\
    &= \lambda u_n+z \sum_{j=1}^{s}a_{ij} k_j
  .\end{align}
  Ahora definimos $K$ como la columna formada por los $k_i$. Es decir,
  \begin{align}
    K = \lambda u_n\bar 1+zAK
  ,\end{align}
  donde $\bar 1$ es una columna cuyas entradas son todas iguales a
  $1$. Despejando $K$, esto es
  \begin{equation}
    K=\lambda (I-zA)^{-1}u_n\bar 1
  .\end{equation}
  Por lo tanto, $F=b^{T}K$ y
  \begin{align}
    u_{n+1}
    &= u_n + hF(t_n,u_n,h,f) \\
    &= u_n + hb^TK \\
    &= u_n + h\lambda b^T(I-zA)^{-1}u_n\bar 1 \\
    &= u_n[1 + h\lambda b^T(I-zA)^{-1}\bar 1]
  .\end{align}
  La función de estabilidad del método es la función
  $R(z)=R(h\lambda)$ dentro del corchete, i.e.
  \begin{equation}
    R(z) = 1 + zb^T(I-zA)^{-1}\bar 1
  .\end{equation}
  
  Debemos mostrar que
   \begin{equation}
    R(z) = \det(I-zA+z\ol 1 b^T)
  .\end{equation}
  La siguiente demostración está sacada de
  \cite[p.213]{butcher2004numerical}.
  Sean $v$ un vector de $n$ entradas y $v'$ el vector de $n-1$
  entradas que se obtiene eliminando la primera entrada de $v$ y
  denotemos como $I'$ a la matríz identidad de $(n-1)\times(n-1)$.
  También denotemos $\ol 1'$ como el vector de $n-1$ entradas formado
  por unos. Entonces la matríz $\ol 1v^{T}$ tiene polinomio
  característico
  \begin{align}
    \chi(x)
    &= 
    \det(xI - \ol 1 v^T) \\
    &= \det
    \begin{bmatrix}
      x-v_1 & -v_2 & -v_3 & \cdots & -v_n \\
      -v_1 & x-v_2 & -v_3 & \cdots & -v_n \\
      -v_1 & -v_2 & x-v_3 & \cdots & -v_n \\
      \vdots & \vdots & \vdots & \ddots & \vdots \\
      -v_1 & -v_2 & -v_3 & \cdots & x-v_n \\
    \end{bmatrix}
    \\
    &= \det
    \begin{bmatrix}
      x-v_1 & -v_2 & -v_3 & \cdots & -v_n \\
      -x & x & 0 & \cdots & 0 \\
      -x & 0 & x & \cdots & 0 \\
      \vdots & \vdots & \vdots & \ddots & \vdots \\
      -x & 0 & 0 & \cdots & x \\
    \end{bmatrix}
    \\
    &= \det
    \left[
      \begin{array}{c|c}
        x-v_1 & -v'^T \\
        \hline \\[-3mm]
        -x\ol 1' & xI'
      \end{array}
  \right]
    \\
    &= [(x-v_1)-(v'^T)(xI')^{-1}(x\ol 1')]\det[xI'] \\
    &= [x-v_1-v'^T\ol 1']\det[xI'] \\
    &= [x-v^{T}\ol 1]\det[xI'] \\
    &= x^{n-1}(x-v^{T}\ol 1)
  .\end{align}
  Luego, el polinomio característico de $I+\ol 1 v^{T}$ es
  \begin{align}
    \det(xI-(I+\ol 1v^{T}))
    &= \det((x-1)I-\ol 1v^{T}) \\
    &= \chi(x-1) \\
    &= (x-1)^{n-1}(x-1-v^{T}\ol 1)
  ,\end{align}
  por lo cual el producto de las raíces es el determinante de $I+\ol
  1v^{T}$
  \begin{equation}
    \det(I+\ol 1v^{T}) = 1+v^{T}\ol 1
  .\end{equation}
  Para $v=z(I-zA)^{-T}b$, tenemos $v^{T}=zb^{T}(I-zA)^{-1}$, así que
  \begin{equation}
    \det(I+z\ol 1b^{T}(I-zA)^{-1}) = 1+zb^{T}(I-zA)^{-1}\ol 1
  ,\end{equation}
  pero el lado derecho es $R(z)$, así que
  \begin{align}
    R(z)
    &= \det(I+z\ol 1b^{T}(I-zA)^{-1}) \\
    &= \frac{ \det(I+z\ol 1b^{T}(I-zA)^{-1})\det(I-zA)}{\det(I-zA)} \\
    &= \frac{ \det((I-zA)+z\ol 1b^{T})}{\det(I-zA)}
  .\end{align}
  En particular, si $A$ es estrictamente triangular inferior,
  $\det(I-zA)=1$, así que
  \begin{equation}
    R(z) = \det((I-zA)+z\ol 1b^{T})
  ,\end{equation}
  como se quería.
\end{sol}

\begin{exercise}
\end{exercise}
\begin{sol}
\end{sol}

\begin{exercise}
\end{exercise}
\begin{sol}
\end{sol}

\end{document}
