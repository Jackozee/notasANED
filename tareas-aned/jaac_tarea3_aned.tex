\documentclass[11pt]{article}
\usepackage[left=3.5cm,right=3.5cm,top=2.5cm,bottom=2.5cm]{geometry}
%\usepackage[spanish]{babel}
\overfullrule = 5mm

\usepackage{amsmath,amsfonts,amsthm}
\usepackage{enumitem,mathtools,graphicx}
\setenumerate[0]{label=(\alph*)}

\newtheorem{definition}{Definición}
\newtheorem{exercise}{Ejercicio}
\newtheorem*{sol}{Solución}
\newtheorem*{theorem}{Teorema}

\newcommand\N{\mathbb N}
\newcommand\R{\mathbb R}
\newcommand\C{\mathbb C}

\title{Análisis numérico para ecuaciones diferenciales \\
Tarea 3 - Métodos Runge-Kutta}
\author{Jorge Alfredo Álvarez Contreras}

\begin{document}
\maketitle

\begin{exercise}
   Programar el método Runge-Kutta Fehlberg para resolver el problema
   de valores iniciales
   \begin{equation}
     y'
     =
     (1-2x)y,
     \quad y(0)=1,
     \quad 0\leq x\leq 4
   ,\end{equation}
   tomando $h_\mathrm{max}=0.1$ y tolerancia $0.01$. Elaborar una
   gráfica del error y una gráfica de $x_n$ contra $h_n$. (Marque con
   un punto los pasos aceptados y con un asterisco los pasos
   rechazados).
\end{exercise}
\begin{sol}
\end{sol}

\begin{exercise}
  Considere la familia de métodos de Runge-Kutta correspondientes al
  table\-ro de Butcher
  \begin{equation}
    \begin{array}{c|cc}
      0 & 0 & 0 \\
      \frac{1}{2\theta} & \frac{1}{2\theta} & 0 \\
      \hline
                        & 1-\theta & \theta
    \end{array}
  .\end{equation}
\end{exercise}
\begin{sol}
\end{sol}

\begin{exercise}
\end{exercise}
\begin{sol}
\end{sol}

\begin{exercise}
\end{exercise}
\begin{sol}
\end{sol}

\begin{exercise}
\end{exercise}
\begin{sol}
\end{sol}

\end{document}
